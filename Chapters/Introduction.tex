\chapter{Introduction}

\section{Background to the study}
The demand for electrical energy is continuously increasing due to rapid industrialization, population growth, and technological advancement. Efficient management of power generation and distribution is therefore critical for maintaining the stability and reliability of modern power systems. One of the most important tools to achieve this goal is load forecasting, which enables power providers predict future electricity demand and plan generation, distribution, and market operations accordingly.

Load forecasting can be categorized into short, medium and long term forecasts depending on their prediction horizon, ranging from a few minutes to years. Among these short term load forecasting is particularly important as it directly impacts the daily operational efficiency of the grid, therefore impacting decisions related to load scheduling, energy decisions and system security.

Traditional forecasting methods such as moving averages, exponential smoothing, and auto-regressive models have long been used due to their simplicity. However they struggle to capture nonlinear and dynamic patterns that are inherent in electricity demand data, which is influenced by factors including weather, time and human activity. As a result of this machine learning (ML) and deep learning (DL) methods have recently gained prominence in recent years as the future of forecasting due to their ability to learn complex relationships within time series data.

In this study the focus is on improving the accuracy of short term load forecasting through the application and comparison of advanced ML and DL models, such as Long-Short Term Memory (LSTM) and Deep Belief Networks(DBNs). These models will be compared with conventional statistical and standard Artificial Neural Networks to evaluate their relative advantages and limitations in the context of STLF.
\section{Objectives of this study}
\subsection{Problems to be investigated}
The main problem to be investigates in this study is how to improve accuracy and reliability of STLF using ML and DL techniques. This research aims to evaluate and compare the performance of models such as the LSTM and DBN against conventional and statistical models to determine which models provides the most accurate and efficient prediction of electricity demand 
\subsection{Purpose of the study}
Accurate STLF is essential for ensuring the stability, reliability and cost-effectiveness of modern power systems. Improving accuracy enables utilities to optimize generation schedules, reduce operational costs and minimize energy wastage. This accuracy will also support the integration of renewable energy systems into the already existing systems.
As energy systems move towards smarter and more sustainable operations, developing robust forecasting models using advanced machine learning and deep learning techniques becomes increasingly relevant for achieving efficient energy management and transitioning to intelligent power systems.
\section{Scope and Limitations}
Scope indicates to the reader what has and has not been included in the study. Limitations tell the
reader what factors influenced the study such as sample size, time etc. It is not a section for excuses as
to why your project may or may not have worked.

\section{Plan of development}
Here you tell the reader how your report has been organised and what is included in each
chapter.

{\bf I recommend that you write this section last. You can then tailor it to your report.}
